\errorcontextlines 999999
\usepackage{attachfile2}


\usepackage[ngerman]{babel}
\PassOptionsToPackage{vlined,linesnumberedhidden}{algorithm2e}
\usepackage[%
    sopra-listings={encoding,cpalette,highlights},%
    sopra-tables, color-palettes={addons},%
    lecture-bibliography={biber,style=numeric-comp},%
    util, lithie-boxes={germanenv,koma,overwrite},%
    lithie-task-boxes={cpalette}, lecture-links={patchurl},%
    lecture-personal-resize,
    lecture-registers={disable}% would interfere with beamer
]{lithie-util}

\usepackage{forest}
\usetikzlibrary{overlay-beamer-styles}

\makeatletter
\sol@list@define@styles{%
  {keywordA: \@declaredcolor{sol@colors@lst@keywordA}\bfseries},%
  {keywordC: \@declaredcolor{sol@colors@lst@keywordC}\bfseries},%
}
\makeatother

\RestyleAlgo{plain}
\def\sbfamily{\fontseries{sb}\selectfont}
\def\textsb#1{{\sbfamily#1}}
\SetKwSty{textsb}
\SetKw{KwRet}{ret}
\SetKwProg{Function}{def}{:}{}
\SetKwIF{If}{ElseIf}{Else}{if}{:}{elif:}{else:}{}
\SetKwFor{While}{while}{:}{}
\lstset{lineskip=5.5pt}
\AtBeginDocument{\def\O{\mathcal{O}}}
\lstfs{10}

\makeatletter
\def\SidebarFile#1#2{%
    \appto\btdl@sidebar@content{%
        {\fontsize{6}{6}\faFileCodeO\qquad\scalebox{1.25}{\textattachfile{#1}{#2}}}\par
    }%
}

\DefinePalette{InplaceMerge}
{Blau,bläulich: RGB(0, 114, 187)}
{Orange,orangenfarben: RGB(255, 131, 53)}% Cadmium orange
{Grün,grünlich: RGB(123, 181, 46)}
{Gelb,gelblich: RGB(255, 200, 87)}
\SetShadeContrast{45}
\UsePalette{InplaceMerge}

\attachfilesetup{%
    author={Florian Sihler},%
    color=darkgray, icon=Tag, mimetype=text/plain%
}

\usetheme[libs,nofootfade,centerfoot]{dividing-lines}
\SetColorProfile*{paletteA}{paletteB}{paletteC}

\usetikzlibrary{arrows.meta,decorations,decorations.pathreplacing,shapes.multipart,tikzmark,shapes.symbols}

\def\info#1{\bgroup\scriptsize\textcolor{gray}{(#1)}\egroup}
\SetAllLinkStyle{#1}
\def\fillfont{\def\mdseries@sf{medium}\sffamily}
\colorlet{lgray}{lightgray!48!white}
\tikzset{
    ldesc/.style={gray,font=\sffamily\sbfamily},
    lrel/.style={fill=white,rounded corners,minimum width=28mm,minimum height=7.5mm,align=center},
    lrel2/.style={fill=white,rounded corners,minimum width=28mm,minimum height=7.5mm*2,align=center},
    lsf/.style={fill=white,rounded corners,minimum width=28mm,minimum height=7.5mm*2,align=center,
        rectangle split, rectangle split parts=2},
    blob/.style={circle,draw, minimum size=1.9em,align=center},
    lblob/.style={blob,writ,font=\fillfont,#1},%
    lblob/.default={fill=lgray, draw=lgray,text=black},
    every picture/.append style={line join=round,line cap=round},%
    every node/.append style={font=\sffamily},%
    lblock/.style={block,writ,font=\fillfont,#1},%
    lblock/.default={fill=lgray, draw=lgray,text=black},
    block/.style={rectangle,draw, align=center, minimum height=1.25em,rounded corners=1.55pt},%
}

\newcommand\parallelcontent[3][t]{%
    \begin{columns}[#1]
    \begin{column}{.475\linewidth}#2\end{column}\hfill
    \begin{column}{.475\linewidth}#3\end{column}
    \end{columns}
}

\usepackage[glows]{tikzpingus}
\usetikzlibrary{decorations.text,graphs}
\hypersetup{colorlinks=false}

\title{In-Place Merge Sort}
\subtitle{Kompromisse, Lügen und Derivate.}
\institute{SP, Universität Ulm}

\author{Florian Sihler}
\email{florian.sihler@uni-ulm.de}

\date{\today}
\outro{Ulm, \today}
\license[]{All rights reserved}

\def\PreTitlepage{\begingroup%
\let\oldinserttitle\inserttitle% allow it to be white on second slide
\let\oldinsertsubtitle\insertsubtitle% allow it to be white on second slide
\colorlet{PINGU@WHITE}{pingu@white}% hacksies for the whites
\only<2|handout:2>{\def\inserttitle{\color{pingu@white}\oldinserttitle}\def\insertsubtitle{\textcolor{pingu@white}{\oldinsertsubtitle}}}%
\onslide<2|handout:2>{%
\savebox0{\tikz{\pingu[lightsaber left=paletteC!90!white,left item angle=-20,eyes shiny,right wing grab,tie=paletteA!90!black,body=paletteA!20!black,lightsaber left length=2.35cm,lightsaber left outer glow factor=.09,cake-hat]}}%
\begin{tikzpicture}[overlay,remember picture]%
    % beamer does not support changes of full background easily. so we do hacksies
    \pgfonlayer{background}
    \path[fill,black!99] (current page.north west) rectangle (current page.south east);
    \endpgfonlayer
    \node[above right=.15cm,scale=.65,xshift=.15cm] (pingu) at(current page.south west) {\usebox0};
    \node[below right,pingu@white,text width=.6\paperwidth,align=flush left] at(pingu.north east){Officially supported by the Ping-Ping-Uh-Uh-\allowbreak Foundation for Emotional Peeps. Peep n' peace and wa\ldots\ oh look, there's cake!};
\end{tikzpicture}}%
}
\def\PostTitlepage{\endgroup}

\addbibresource{./references.bib}

\setcounter{tocdepth}{4}
\newsavebox\pinguA \newsavebox\pinguB \newsavebox\pinguC \newsavebox\pinguD
\newcommand*\mb[1]{$\underset{\text{\color{shadeD}\T{\strut\Large\textbullet}}}{\text{#1}}$}

\usepackage[link]{qrcode}
\outroright{\smash{\raisebox{1.33cm/2}{\qrcode[height=1.33cm]{https://github.com/EagleoutIce/Episode-Inplace}}}\begin{tikzpicture}[remember picture,overlay]
    \node[above left,btdl@color@white,scale=.475] at (current page.south east) {\href{https://github.com/EagleoutIce/Episode-Inplace}{Slides and \LaTeX-sources on GitHub!}};
\end{tikzpicture}}

\def\treewidth{\linewidth}
\newcommand*\tree[2][]{\downsize{\treewidth}{\begin{forest}for tree={lblob,minimum width=2.5em,s sep=4em-level*.5em,edge={line width=3pt,lgray,line cap=butt},#1}#2\end{forest}}}
\forestset{w/.style={fill=shadeGray,draw=shadeGray,text opacity=.5,edge={shadeGray}},m/.style={fill=shadeA,draw=shadeA,edge={shadeGray}},g/.style={fill=shadeA,draw=shadeA}}%

\hfuzz=60cm
\makeatletter
\def\mtx@blk{X}
\def\mtx@k{6mm}
% calculate the number of steps required to stay visible
\mode<presentation>{
\def\mtx@split@gol#1|handout:#2\@nil{\the\numexpr#1-\beamer@slideinframe\relax}
\def\mtx@split@tar#1|handout:#2\@nil{}
}\mode<handout>{% TODO: for handout, subtract handout trigger ids
\def\mtx@split@gol#1|handout:#2\@nil{%
    \ifnum\beamer@slideinframe=1
    \the\numexpr#1-\triga\relax
    \else\ifnum\beamer@slideinframe=2
    \the\numexpr#1-\trigb\relax
    \else\ifnum\beamer@slideinframe=3
    \the\numexpr#1-\trigc\relax
    \else\ifnum\beamer@slideinframe=4
    \the\numexpr#1-\trigd\relax
    \else\ifnum\beamer@slideinframe=5
    \the\numexpr#1-\trige\relax
    \ifnum\beamer@slideinframe=6
    \the\numexpr#1-\trigf\relax
    \fi\fi\fi\fi\fi\fi
}
\let\mtx@split@tar\mtx@split@gol
}
\tikzset{MTX@BLK/.style={rectangle,minimum size=\mtx@k,inner sep=\z@,outer sep=\z@,fill=#1,draw=gray!80!white,line width=.75\p@}}
\def\Matrix#1#2#3#4#5#6#7#8#9{%
\def\triga{#4}\def\trigb{#5}\def\trigc{#6}\def\trigd{#7}\def\trige{#8}\def\trigf{#9}
\scope[scale=1.15,every node/.append style={transform shape}]
\scope
\clip[rounded corners=2\p@] (.5*\mtx@k, -.5*\mtx@k) rectangle ++(#2*\mtx@k, -#3*\mtx@k);
\foreach[count=\y] \line in {#1} {
    \foreach[count=\x] \tar/\gol/\ded in \line {
        \ifx\tar\mtx@blk
            \node[MTX@BLK=gray!38!white] (\x-\y) at (\x*\mtx@k, -\y*\mtx@k) {};
        \else
            \edef\mtx@tmp{\gol}%
            % overdraw me baby
            \node[MTX@BLK=white,text=lightgray] (\x-\y) at (\x*\mtx@k, -\y*\mtx@k) {\footnotesize\expandafter\mtx@split@tar\mtx@tmp\@nil};
            \only<\tar>{
                \node[MTX@BLK=shadeB,text=gray] (\x-\y) at (\x*\mtx@k, -\y*\mtx@k) {\footnotesize\expandafter\mtx@split@gol\mtx@tmp\@nil};
            }
            \only<\gol>{
                \node[MTX@BLK=shadeA] (\x-\y) at (\x*\mtx@k, -\y*\mtx@k) {};
            }
            \only<\ded>{
                \node[MTX@BLK=shadeGray] (\x-\y) at (\x*\mtx@k, -\y*\mtx@k) {\T{\color{gray!80!white}\textbullet}};
            }
        \fi\
    }
}
\endscope
\draw[rounded corners=2\p@,gray!80!white,line width=1.65\p@] (.5*\mtx@k, -.5*\mtx@k) rectangle ++(#2*\mtx@k, -#3*\mtx@k);
\endscope}
\makeatother